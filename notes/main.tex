\documentclass[11pt]{article}

\usepackage{amsmath}
\usepackage{todonotes}
%--------------------------------------
\usepackage[hidelinks]{hyperref}
\usepackage{xcolor}
\usepackage{geometry,tabularx,fancyhdr}
\title{Finding Lexicographically Smallest Isomorphic Model by SAT}
\author{Mikol\'a\v{s} Janota}
\newcommand{\xpi}[2]{x_{{#1}\rightarrow{#2}}} 
\usepackage[ruled,vlined]{algorithm2e}
\DontPrintSemicolon%
\SetKwFunction{SAT}{SAT}

\begin{document}
\maketitle

\section{Encoding}

Given an $f$ on $D^2\rightarrow D$ with $D=1..n$ we are looking for an $f'$ that is
isomorphic to $f$ and is lexicographically smallest. We use repeated calls
to a SAT solver to calculate $f'$.

The crucial part of the encoding is to decide if there is an $f'$ isomorphic to
$f$ such that a specific value can be placed in a specific position of the
multiplication table of $f'$. We denote this requirement as follows.

\begin{equation}\label{eq:given}
  f'(i,j)\mapsto k, \text{ for some  given } i,j,k\in D
\end{equation}

Equation~\eqref{eq:given}  is equivalent to deciding the existence of a permutation $\pi$ such that

\begin{equation}\label{eq:pi}
  \pi\left(f(\pi^{-1}(i),\pi^{-1}(j))\right) = k
\end{equation}

To translate to SAT, introduce Boolean variables $\xpi{d}{r}$ representing
whether $\pi(d)=r$, for $d,r\in D$. To ensure that these variables represent a
bijection, introduce the following constraints (cardinality constraints are
converted to CNF by standard means).

\begin{align}
  \sum_{d\in D} \xpi{d}{r}=1, \text{ for } r\in D\\
  \sum_{r\in D} \xpi{d}{r}=1, \text{ for } d\in D
\end{align}

Then, the equality~\eqref{eq:pi} yields the following implications.
\todo{We don't really need this case-split for the presentation but this is how
I implemented it to reduce unnecessary clauses.}

\begin{align}
  \left(\xpi{d_1}{i}\land\xpi{d_2}{j}\right)\Rightarrow\xpi{f(d_1, d_2)}{k}\text{ for } d_1\neq d_2\in D
  \text{ if } i=j\\
  \xpi{d}{i}\Rightarrow\xpi{f(d, d)}{k}\text{ for } d\in D \text{ if } i\neq j
\end{align}

Note that $f$ does not explicitly appear in the encoding as $f(d_1, d_2)$ is evaluated in encoding time.

\subsection{Finding the Model}
To find the lexicographically  smallest model $f'$ isomorphic to $f$ we
gradually fill in the multiplication table of $f'$, trying the smaller values
first as shown in Algorithm~\ref{algorithm:basic}.  It first tries placing $1$
in the top left corner and if that is possible  it moves onto the  next cell. If
it is not possible,  is tries placing 2 in the top left corner, and so forth.
Once  it finds that it is possible to place a certain value in a cell,
it is required to stay there for the rest of the course of the algorithm.

%%%%%%%%%%%%%%%%%%%%%
% File:  basic_algo.tex
% Author: mikolas
% Created on: Fri Dec 30 21:04:29 CET 2022
% Copyright (C) 2022, Mikolas Janota
\begin{algorithm}[t]
  \caption{iterative calls}\label{algorithm:basic}
  %
  $a\gets []$ \tcp*[r]{empty sequence of assignments}
  \For{$i\in 1..|D|$}{
    \For{$j\in 1..|D|$}{
      $k\gets 1$\;
      \lWhile{$\lnot\SAT(a\texttt{ ++ }[f'(i,j)\mapsto k])$}{
        $k\gets k + 1$
      }
      $a\gets a\texttt{ ++ }[f'(i,j)\mapsto k]$\tcp*[r]{extend the sequence}
    }
  }
\end{algorithm}
%%%
%%%%%%%%%%%%%%%%%%%%%


\section{Possible Improvements and Extensions}

Algorithm~\ref{algorithm:basic} suffers from a high number of SAT calls; it is
$\Omega(|D|^3)$ in the worst case (for quasigroups  for instance). Some special
cases can be identified.

 \begin{itemize}
   \item  If every line/column of the given $f$ is a permutation then $f'$ must
     have the same property
   \item The first row of $f'$  must be a permutation of one of the rows $r$
     that maximizes the number of elements $r$.
   \item The top left corner is 1 iff $f$ has at least one idempotent.
     Combining it is with the previous condition  could possibly determine which row gets
     swapped into  the first one.
   \item Once the row of $f'$ is filled in, we can look at  some of its
     characteristics and identify it in $f$.  Such characteristics  could for
     instance be  number of elements $f(i, j)=i$, $f(i, j)=j$, etc.  Maybe it
     has some sort of a well-known name?  Relation to symmetries?
 \end{itemize}

Because the lexicographically  smallest model is unique, it can be used to
decide homomorphism between models. There we don't care in which order the
multiplication table is traversed. This means that we can run
Algorithm~\ref{algorithm:basic} in a completely different order that fixes the
permutation $\pi$ more quickly.


\end{document}
